% Tarea #7
%   -	Estudiante/Autor: 	Miguel E. Cruz Molina
%   - 	# de estudiante: 	801-16-1956
%   - 	Curso y sección: 	MATE 4089-0U1
%   - 	Instructor: 		Prof. Raúl F. Figueroa Guerrero
%   - 	Fecha de última edición: 	3/28/2021

\documentclass[12pt]{article}
\textwidth=6in \textheight=9in \hoffset=-0.375in \voffset=-0.75in

\usepackage{graphicx}
\usepackage{caption}
\usepackage{subcaption}
\usepackage{setspace}
\usepackage{tabularx}
\usepackage{amsmath}
\usepackage{amsfonts}
\usepackage{amssymb}
\usepackage{amsthm}
\usepackage[makeroom]{cancel}
\everymath{\displaystyle}

\graphicspath{ {./images/} }

\setstretch{1.25}

\title{Tarea \#7}
\date{26 de marzo de 2021}
\author{Miguel E. Cruz Molina}

\begin{document}

\begin{tabularx}{.95\textwidth}{lXcXr}
	Miguel E. Cruz Molina && MATE 4089-0U1 && 26 de marzo de 2021\\[.15in]
	(801-16-1956) &&  && Prof. R. Figueroa\\[.15in]
\end{tabularx}

\begin{center}
	\textbf{Tarea \#7}
\end{center}

\begin{enumerate}

\setcounter{enumi}{19}

% Ejercicio #20:

\item Dada la correspondencia $A \leftrightarrow P$, $B \leftrightarrow Q$ y $C \leftrightarrow R$ entre los vértices de $\bigtriangleup ABC$ y los de $\bigtriangleup PQR$, si $\angle A \cong \angle P$ y $\frac{CA}{AB} = \frac{RP}{PQ}$, demuestre que $\bigtriangleup ABC \sim \bigtriangleup PQR$. \\

{\it Demostración}: Note primero que

\begin{center}
$\begin{array}{@{\hskip 0 in}lll}
\frac{CA}{AB} = \frac{RP}{PQ} & \implies & \frac{CA}{RP} = \frac{AB}{PQ} \quad (*) \\
& \\
					& \implies & \frac{CA}{CA-RP} = \frac{AB}{AB-PQ} \\
& \\
					& \implies & \frac{1}{-1}\cdot\frac{CA}{CA-RP} = \frac{1}{-1}\cdot\frac{AB}{AB-PQ} \\
& \\
					& \implies & \frac{CA}{RP-CA} = \frac{AB}{PQ-AB} \quad (**) \\
\end{array}$
\end{center}

Por el Axioma III-1, existen puntos $D \in \overrightarrow{PQ}$ y $E \in \overrightarrow{PR}$ tales que $\overline{AB} \cong \overline{PD}$ y $\overline{AC} = \overline{PE}$. Luego, por el Teo. 3 (LAL), $\bigtriangleup ABC \cong \bigtriangleup PDE$, y, adicionalmente, $P$, $D$ y $Q$ son colineales y $P$, $E$ y $R$ también lo son. De esto, obtenemos dos casos:

\begin{itemize}

\item \underline{Caso \#1}: $D = Q$ o $E = R$. Sin perdida de generalidad, asuma que $D = Q$. Entonces, $\overline{AB} \cong \overline{PE} =\overline{PQ}$, y por $(*)$, tenemos que $1 = \frac{AB}{PQ} = \frac{CA}{RP}$, lo cual implica que $\overline{CA} \cong \overline{RP}$. Por lo tanto, por estas congruencias y $\angle A \cong \angle P$, aplicando al Teo. 3 (LAL), $\bigtriangleup ABC \cong \bigtriangleup PQR$ y $\bigtriangleup ABC \sim \bigtriangleup PQR$. \\

\item \underline{Caso \#2}: $D \neq Q$ y $E \neq R$. Asuma sin perdida de generalidad que $P-D-Q$ y que $P-E-R$ (de otro modo, la demostración prosigue igual pero con $D$ sustituyendo a $Q$ y viceversa o $E$ sustituyendo a $R$ y viceversa). Entonces, por resta de segmentos (Lema 5),  $DQ = PQ - PD = PQ - AB$ y $ER = RP - PE = RP - AC$. Luego, por $(**)$,

\begin{center}
$\frac{AC}{RP-AC} = \frac{AB}{PQ-AB} \implies \frac{AC}{ER} = \frac{AB}{DQ} \implies \frac{PE}{ER} = \frac{PD}{DQ},$
\end{center} 

Por el Teorema 42 y esto último, obtenemos que $\overleftrightarrow{DE} \parallel \overleftrightarrow{QR}$, y, por consecuencia del Teorema 34, $\angle B \cong \angle PDE \cong \angle Q$ y $\angle C \cong \angle PED \cong \angle R$. Finalmente, aplicando el Teorema 43 sobre la paralelidad encontrada,

\begin{center}
$\frac{PD}{PQ} = \frac{PE}{PR} = \frac{DE}{QR} \implies \frac{AB}{PQ} = \frac{AC}{PR} = \frac{BC}{QR}$
\end{center}
 
\end{itemize}

$\therefore\hspace{5 pt} \bigtriangleup ABC \sim \bigtriangleup PQR \quad\qed$ \\

\begin{figure}[h]
	\centering
	\includegraphics[height = 1.8in]{ejercicio20caso2}
	\caption{Caso \#2 del Ejercicio 20: $D \neq Q$ y $E \neq R$}\label{fig:1}		
\end{figure}

% Ejercicio #21:

\item Sea $D$ el pie de la línea perpendicular a la hipotenusa $\overline{AB}$ del triángulo rectángulo $\bigtriangleup ABC$, que pasa por el vértice $C$. Demuestre que $(CD)^2 = AD \times DB$. \\

{\it Demostración}: Note primero que $\angle ADC \cong \angle ACB$, al ambos ser rectos (Teo. 9). Luego, por el Teorema 39, $\angle A + \angle B = 90^{\circ} \hspace{3 pt}\wedge\hspace{3 pt} \angle A + \angle ACD = 90^{\circ} \implies \angle ACD \cong \angle B$. Similarmente, $\angle CDB \cong \angle ACB$, así que $\angle B + \angle A = 90^{\circ} \hspace{3 pt}\wedge\hspace{3 pt} \angle B + \angle BCD = 90^{\circ} \implies \angle A \cong \angle BCD$. Luego, existe una correspondencia entre los vértices de $\bigtriangleup ACD$ y $\bigtriangleup CBD$, $A \leftrightarrow C$, $C \leftrightarrow B$ y $D \leftrightarrow D$, tal que los ángulos internos correspondientes a estos vértices en sus respectivos triángulos son congruentes entre sí. Por lo tanto, aplicando al Teo. 45 (AAA), $\bigtriangleup ACD \sim \bigtriangleup CBD$ y, por definición de semejanza de triángulos,

\begin{center}
$\begin{array}{@{\hskip 0 in}lll}
\frac{AC}{CB} = \frac{CD}{BD} = \frac {DA}{DC} & \implies & \frac{CD}{DB} = \frac{AD}{CD} \\
& \\
							       & \implies & \frac{(CD)^2}{DB} = AD \\
& \\
							       & \implies & (CD)^2 = AD \times DB \quad\quad\qed \\
\end{array}$
\end{center}

% Ejercicio #22:

\item Demuestre que si los lados de un triángulo $ABC$ satisfacen la relación $(AB)^2 = (AC)^2 + (CB)^2$, entonces $\bigtriangleup ABC$ es un triángulo rectángulo de hipotenusa $\overline{AB}$. \\

{\it Demostración}: Considere a un punto arbitrario $R$. Eligiendo a un rayo arbitrario que tenga a $R$ como vértice, el axioma III-1 nos garantiza que podemos trazar un segmento $\overline{RP}$ tal que $\overline{RP} \cong \overline{CA}$. Usando ahora a los axiomas III-1 y III-4, podemos construir otro segmento $\overline{RQ}$ tal que $\overline{RQ} \cong \overline{CB}$ y $\angle PRQ$ es recto. Luego $\bigtriangleup PRQ$ es un triángulo recto con hipotenusa $\overline{PQ}$, y por el Teo. 49 (Pitágoras), $(PQ)^2 = (PR)^2 + (RQ)^2$. Adicionalmente, por las congruencias definitorias de estos segmentos, tenemos que $AC = PR$ y $CB = RQ$. Por lo tanto,

\begin{center}
$\begin{array}{@{\hskip 0 in}lll}
AC = PR \hspace{3 pt}\wedge\hspace{3 pt} CB = RQ & \implies & (AB)^2 = (AC)^2 + (CB)^2 = (PR)^2 + (RQ)^2 = (PQ)^2 \\
& \\
								  & \implies & AB = \sqrt{(AC)^2 + (CB)^2} = \sqrt{(PR)^2 + (RQ)^2} = PQ\\
& \\
								  & \implies & AB = PQ \\ 
& \\
								  & \implies & \frac{AB}{PQ} = \frac{BC}{QR} = \frac{CA}{RP} = 1 \\
& \\
								  & \implies &\bigtriangleup ABC \sim \bigtriangleup PQR 
										\quad(\text{Teo. 47 (LLL)})\\
& \\
								  & \implies & \angle C \cong \angle R = 90^{\circ} 
\end{array}$
\end{center}

$\therefore \hspace{5 pt} \bigtriangleup ABC$ es un triángulo recto con hipotenusa $\overline{AB}. \quad\quad\qed$ \\

\pagebreak

% Ejercicio #23:

\item Demuestre que, si dos cuerdas $\overline{AD}$ y $\overline{BC}$ de una circunferencia se intersecan en un punto $E$,entonces $AE \cdot ED = BE \cdot EC$. \\

{\it Demostración}: Considere los triángulos $\bigtriangleup ACE$ y $\bigtriangleup BDE$. Claramente, $\angle AEC \cong \angle BED$, puesto que ambos son opuestos (Teo. 7). Adicionalmente, como $\angle ACE = \angle ACB$ $(C - E - B)$ y $\angle BDE = \angle BDA$ $(D - E - A)$, ambos ángulos $C$ y $D$ interceptan al mismo arco en la circunferencia (de $A$ a $B$). Por ende, por el Teorema 39, $\angle ACE \cong \angle BDE$. Similarmente, como $\angle CAE = \angle CAD$ $(A - E - D)$ y $\angle DBE = \angle DBC$ $(B - E - C)$, ambos ángulos $A$ y $B$ interceptan al mismo arco en la circunferencia, el de $C$ a $D$, y $\angle CAE \cong \angle DBE$. Estas tres congruencias angulares significan, por el Teo. 45 (AAA), que $\bigtriangleup ACE \sim \bigtriangleup BDE$, y por lo tanto,

\begin{center}
$\frac{AC}{BD} = \frac{CE}{DE} = \frac{AE}{BE} \implies \frac{AE}{BE} = \frac{EC}{ED} \implies AE \cdot ED = BE \cdot EC \hspace{5 pt}\qed$
\end{center}

\begin{figure}[h]
	\centering
	\includegraphics[height = 2.5in]{ejercicio23}
	\caption{Ejercicio 23: Circunferencia con triángulos $\bigtriangleup ACE$ y $\bigtriangleup BDE$}\label{fig:2}		
\end{figure}
 
\end{enumerate}

\noindent

\end{document}